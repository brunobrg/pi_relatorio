\chapter[Conclusões]{Conclusões}

  \section{Contribuições do trabalho}

  A gerência, o grupo de integração e os subgerentes contribuíram com a determinação do escopo macro e EAP geral, organização do grupo geral e indicação de tarefas. Os grupos de planejamento de informações e conscientização, coleta, separação e tratamento de lixo seco e tratamento de lixo orgânico, contribuíram com o desenvolvimento dos pontos mais específicos do escopo e a EAP específica de cada grupo. A gerência desenvolveu o primeiro relatório e apresentação do ponto de controle 1.

  \section{Limitação das soluções propostas}

  As soluções propostas são limitadas pelas características do setor/área escolhida para desenvolvimento do projeto. O Setor Central do Gama tem uma grande circulação de pessoas por dia, além de possuir áreas de moradias, comercial, hospitalar e escolar às quais o projeto atual planeja soluções eficientes para a coleta e destinação do lixo nessas áreas específicas.

  \section{Dificuldades encontradas}

  As maiores dificuldades encontradas no desenvolvimento do projeto até o presente momento é o tempo escasso. Logo, devemos planejar um aproveitamento otimizado do tempo em que temos nos encontros e do tempo que temos para cada entrega.

  \section{Trabalhos futuros}

    Agora que já temos os limites do projeto, escopo e EAP, é planejado desenvolvermos um estudo teórico das propostas de soluções e o desenvolvimento de uma solução otimizada e eficiente.
