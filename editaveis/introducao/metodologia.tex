\section{Metodologia}

  Para o gerenciamento do projeto, a gerência adotou práticas do \textit{Project Management Body of Knowledge}
  (PMBoK) para auxiliar na organização do projeto e caracterização do produto.
  O PMBoK é um conjunto de boas práticas de gerenciamento de projeto, e principal modelo
  utilizados por profissionais. O guia PMBoK divide o projeto em 5 grupos de processos e
  9 áreas de conhecimento, fornecendo diretrizes de o que deve ser relizado relevante
  a cada área de conhecimento em cada etapa do ciclo de vida do projeto.

  Nesta primeira etapa do projeto foi realizado o grupo de
  processo de iniciação com a criação do termo de abertura
  do projeto (TAP), e iniciado o grupo de processos de planejamento
  com a criação da estrutura analítica do projeto (EAP) preliminar, da
  declaração do escopo e do cronograma preliminar.

  O Termo de abertura do projeto é o documento que autoriza formalmente a existência
  do projeto e serve de guia para o desenvolvimento por possuir aspectos gerais
  pertinentes ao projeto e ao produto. Ele alimenta informações para a criação dos planos
   auxiliares, referentes ao grupo de processo de planejamento, e outros artefatos do projeto, como
   a estrutura analítica do projeto.
   A EAP é um modelo hierarquico orientado aos pacotes de entrega do projeto. Sua formulação
   ajuda a organização a ter uma visão do todo a ser produzido, e facilita a comunicação
   entre equipe e cliente. Dos pacotes de entrega identificados na EAP, gera-se o cronograma, que lista
   todas as atividades que devem ser executadas, seus responsáveis e prazos, para que os gerentes realizem
   um monitoramento efetivo do projeto e possa controlar os recursos para o sucesso do projeto.  
