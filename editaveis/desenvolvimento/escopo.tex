\section{Escopo}

O presente projeto pretende elaborar um sistema inteligente
 e eficiente de coleta, separação e destinação do lixo
 urbano no Setor Central do Gama – DF. Foi escolhido este
  setor por ser um local de grande circulação de pessoas,
  com várias escolas, rodoviária, shopping center e grande
  comércio, e por ser um local de maior conhecimento dos membros
  do grupo desenvolvedor do projeto.

  O processo estudado para o desenvolvimento do presente projeto
   consiste basicamente de quatro partes: conscientização e comunicação,
   coleta, separação e destinação.

   \subsection{Planejamento da comunicação e conscientização}

   \subsubsection{Comunicação}

   Toda informação gerada em um processo pode ser reutilizada para gerenciamento, otimização, planejamento e execução de um plano de ação mais eficaz. Para haver uma transmissão de todas essas informações deve ser estabelecido um canal de comunicação entre os dois lados do sistema, isto é, uma central em que se localizasse os dados para avaliação. Portanto, o mapeamento financeiro, situação em que se encontra os sistemas de geração de energia e reciclagem e os gastos com campanhas de conscientização, devem ser registradas em uma central de informações disponíveis para a gerência da empresa.

   Assim como o projeto inteiro do Sistema de Coleta e Destinação, as informações são dividas e classificadas por etapas, sendo: dados sobre a coleta; informações da produção sobre a geração de energia e da reciclagem dos materiais; e os custos de campanhas, treinamentos e outros gastos relacionados à conscientização do público. Todas essas informações devem ser passadas periodicamente através de um meio de comunicação entre as equipes responsáveis pela coleta, destinação e conscientização para que os responsáveis pela coordenação do projeto possam trabalhar em um plano de ação acerca dos dados coletados, sabendo onde atacar os problemas ou ineficiências.


   Para as informações sobre a coleta, é esperado que a manutenção dos caminhões e das lixeiras e com a mão-de-obra responsável pelo manuseio e manutenção desses equipamentos, sejam coletadas e registradas. Todas essas ações geram custos, e é importante contabilizar todos eles para o efetivo controle do projeto.

   No processo de destinação de resíduos (Reciclagem e Geração de Energia), têm de ser avaliados aspectos como a eficácia dos equipamentos através de dados acerca da produção do biogás, quanto será gasto com manutenção e mão-de-obra de tais máquinas, se o preço de custo está compensando em relação ao preço de mercado. Esses pontos serão controlados e observados para obter um maior aproveitamento do processo de reciclagem e torná-lo rentável e com um alto custo benefício. A parte de manutenção do maquinário é extremamente importante para manter o processo sempre funcionando e ativo. É nessa seção também espera que o produto reciclado gere uma receita para a empresa, tendo em vista um sistema sustentável, sem necessidade de verbas externas à longo prazo.

   Na área da conscientização serão necessários investimentos nas áreas de divulgação das campanhas para realização dos eventos e gastos nos treinamentos dos profissionais. Todos esses gastos são importantes para o funcionamento do projeto, devido aos seguintes motivos: falta de mão-de-obra qualificada, obtenção de novos usuários ao sistema de coleta seletiva, identificação do interesse do público no projeto, apresentação dos trabalhos realizados nas escolas onde são feitas as atividades sobre coleta seletiva, além de reforçar a fidelização dos usuários ativos.

   \subsubsection{Conscientização}

   Com a atual situação do mundo acerca do tema sustentabilidade, a coleta seletiva se tornou uma ação cada vez mais próxima do cotidiano das pessoas. Esse processo de coleta seletiva requer conhecimentos acerca do meio ambiente e da interferência do homem neste meio. Para tal, é necessário um trabalho voltado para a conscientização da população acerca do sistema de coleta e destinação.

   A conscientização da população deve atingir tanto a população mais jovem, quanto a população mais velha, tendo em vista atingir ambas as faixas etárias, foi elaborado um plano de ação dividido em 5 partes principais: Programa de Instrução sobre Coleta Seletiva (PICS), Eventos Comunitários, Cartilha Educacional, Mídia e Treinamento Profissional.

   \subsubsection{Canais de comunicação}

   A mídia se torna um canal de comunicação com a população, isto é, devem ser elaborados meios nos quais a população fique informada sobre os dados da coleta, como horários e locais, estatísticas de quanta energia está sendo gerada, quanto lixo está sendo reciclado, informações sobre os eventos e sobre como deve ser realizada a coleta.

   Foi pensado então na formulação de um site ou um aplicativo em que fornecesse toda a base de informações para o público, sobre o método de coleta para lixo seco, lixo orgânico, o descarte correto do óleo de cozinha, os locais no qual os contêineres poderiam estar, os horários em que os caminhões de coleta passarão e outras coisas a mais.

   A promoção da campanha também pode ocorrer pela cobertura nos locais dos eventos comunitários, divulgação das datas e horas em que começam tais eventos e sobre o site em que as pessoas encontrariam informações.

   \subsubsection{Programa Educacional (PICS)}

   A elaboração de programas educacionais para instrução sobre a coleta seletiva, dada por educadores ambientais é uma ideia que visa atingir as escolas, procurando conscientizar as crianças e pré-adolescentes sobre como deve ser feito o descarte correto do lixo e qual o impacto disto no meio ambiente. A ideia inicial seria de realizar 8 aulas, sendo 4 como palestras e mais 4 como desenvolvimento de alguma atividade proposta pelo educador ambiental.

   As diretrizes desse programa seria:

   \begin{enumerate}
     \item Explicar quais os problemas do descarte incorreto d lixo;
     \item Demonstrar os benefícios da coleta seletiva;
     \item Compartilhar experiências pessoais de profissionais da área;
     \item Explicar como funciona o sistema de coleta;
     \item Promover um trabalho educativo e integrador.
   \end{enumerate}

   \subsubsection{Eventos comunitários}

   Os eventos comunitários são ideias propostas para atingir a população como um todo. Atividades desenvolvidas por alunos da rede pública poderão ser apresentadas à população, os pais destas crianças poderiam participar de workshops ou de palestras com especialistas na área. Inicialmente esses eventos ocorreriam duas vezes ao semestre, de forma a não saturar a população de informações e dar tempo para o desenvolvimento de alguns trabalhos.

   Esses tipos de eventos visam expandir os conceitos de coleta seletiva e aproximar a comunidade de como está acontecendo o trabalho. Trabalhos feitos a partir dos materiais reciclados podem demonstrar o poderio que a reciclagem tem, e, a longo prazo, incentivar os ideais de sustentabilidade.

   \subsubsection{Cartilha educativa}

   Uma cartilha com conteúdo suficiente que tente incorporar as diretrizes definidas no Programa Nacional de Educação Ambiental (PRONEA). Ela deve promover o conhecimento sobre como conscientizar as pessoas de como o meio ambiente é afetado ao descartar o lixo sem tomar conhecimento do seu destino, sem tomar os devidos cuidados. É importante que esta reflexão seja feita e a destinação seja conhecida, ou seja, deve haver informações de como o lixo deve ser coletado, tratado e disposto de forma a não poluir e degradar o meio ambiente e não gerar impactos sobre a saúde humana.

   Outra cartilha independente, voltada para o público infantil, também deveria ser desenvolvida para que seja usada no PICS. Esta cartilha deveria envolver assuntos mais práticos e conceitos básicos, como a ideia dos 3R’s (Reciclar, Reutilizar e Reduzir), explicar quais os tipos de lixo e instruir as crianças à coleta.

   \subsubsection{Treinamentos profissionais}

   Todo a mão de obra envolvida no processo de coleta seletiva requer profissionais que conheçam toda a cadeia de coleta, reciclagem e produção. Especialmente os coletores de lixo necessitam de cursos que forneçam competências e habilidades para uma autogestão das cooperativas. Portanto, incluir cursos sobre desenvolvimento sustentável, segurança do trabalho, gestão de resíduos, gerenciamento, legislação, informática e empreendedorismo, seriam temas base para a construção de um treinamento para a mão de obra por inteiro.

   O curso deve ser oferecido por uma mão de obra qualificada, para isto, educadores ambientais são os mais indicados para falar sobre os temas de sustentabilidade e coleta seletivo. Os profissionais mais indicados para ministrar os cursos sobre os diversos campos da coleta seletiva, são: gestor ambiental, técnicos em meio ambiente e segurança, engenheiro ambiental, engenheiro químico de meio ambiente, economista ambiental, advogado ambiental, professores e pesquisadores na área de meio ambiente. Os profissionais citados possuem formação nas áreas propostas como temas base dos treinamentos. Esses conhecimentos são importantes para uma gestão autônoma dos cooperados e associados das cooperativas de coleta seletiva.


  \subsection{Planejamento da coleta}

    O objetivo do plano de coleta é o estudo de rotas otimizadas para recolherem mais lixos e armazená-los nos containers corretos. Os lixos separados nos containers serão então recolhidos de acordo com seu conteúdo, e encaminhados para a central de separação e tratamento.

    Os containers usados serão primeiramente os que já existem em condomínios e estabelecimentos comerciais que possuem a separação entre lixo orgânico e seco. Em áreas residenciais sem lugar específico para depositar o lixo, containers de metal ou plástico serão instalados nas ruas para que os moradores possam colocar o lixo separando orgânico do seco.

    Caminhões de coleta serão separados conforme sua rota e produto recolhido. Caminhão com garras terão rotas para vias comerciais que possuem containers na calçada com lugar reservado para o caminhão estacionar e usar as garras mecânicas no recolhimento do lixo. Caminhão com recolhedores manuais, serão projetados para áreas residenciais que receberão os containers para a coleta de lixo, pois nesses lugares não se pode garantir que 100\% dos moradores colaborarão colocando seu lixo nos containers e as vias podem estar congestionadas com carros. Caminhão para a coleta de óleo terá recipientes especiais para armazená-los de forma segura e pessoas irão fazer a coleta, indo nas residências e estabelecimentos comerciais cadastradas e irão recolher para fazer a destinação correta.

  \subsection{Planejamento da separação e destinação do lixo}

  \subsubsection{Porque o serviço de destinação e separação}

  No Distrito Federal o lixo gerado é depositado em aterros sem nenhum tratamento prévio. Tendo em média, de acordo com a BELACAP num estudo feito no ano 2000, que cada pessoa gera 1kg de lixo diariamente no DF, e em tal época a população do DF era de aproximadamente 1.981.933 habitantes [1], é claramente visível que tal medida é se torna inviável devido ao crescimento da população, que atualmente, num estudo divulgado pelo IBGE em 28/05/2015, é de aproximadamente 2.914.830 [2] e que num. Por estes motivos novas destinações e tipos de tratamento de lixo devem ser adotadas.

  \subsubsection{Proposta}

    Como apenas 16\% do DF realiza coleta seletiva, onde o lixo seco é separado do orgânico, segundo o SLU numa reportagem feita pelo correio Braziliense [3], há uma grande necessidade de empresas que separem o lixo destinado aos aterros para que ele se possa tratado. E tendo em vista isso e que atualmente a maioria das empresas compram materiais como metal e papel e reutilizam como matéria prima há também a possibilidade de gerar lucro com tal tipo de tratamento, viabilizando tal serviço. Então considerando tal propostas os materiais mais abundantes encontrados nos lixos e que serão trabalhados nesta parte do projeto, são:
    \begin{itemize}
      \item Vidros;
      \item Metais em geral;
      \item Papeis;
      \item Plásticos;
      \item Lixos eletrônicos.
    \end{itemize}

    O objetivo é que o lixo recebido seja devidamente separado entre seco e orgânico, por um maquinário especifico, e um pessoal devidamente treinado, em um galpão onde o lixo será recebido e estocado após a separação. E então será tratado de acordo com cada tipo de material e destinado a empresas terceiras que comprem tais materiais.

    Então equipes dentro desta área do projeto terão que avaliar e definir as formas de
    separação, os tipos de lixo e a destinação. Para tal, envolve a contratação de pessoal, maquinas seletoras de lixo,
    locação de espaço para trabalho, custo relacionado ao pessoal, local e maquinário.
    Dentre os tipos de lixo, eles podem ser não utilizável, que são destinados a compactação e aterro,
    e os aproveitável, como o metal, plástico, vidro, papéis e eletrônicos, que podem ser tratados, caso
    necessário, e vendidos ou reutilizáveis.

  \subsection{Planejamento do tratamento de resíduos orgânicos}

  Sabendo que em 2015 foram coletadas 2.878t/d de lixo no Distrito Federal (DF), segundo o Relatório Anual do Sistema de Limpeza Urbana (SLU) de 2015, nota-se a grande quantidade de lixo urbano que é produzido no DF, tendo como base esse dado observa-se o grande problema que é a quantidade enorme de lixo urbano produzido no país. Na busca por meios de reutilizar e reaproveitar a parte orgânica desse lixo, para diminuir a quantidade de lixo que são destinados para lixões ou que não possui uma destinação correta, e para prolonga o ciclo de vida desses rejeitos, reduzindo a utilização de matérias primas retiradas diretamente de seus berços, diminuindo a emissão de gases poluentes na atmosfera, incentivo a educação ecológica e a boas práticas, geração de empregos e créditos de carbono (Petrochi-Slide).

  Propõe-se o tratamento do lixo orgânico para gerar biogás, fertilizante e efluentes mineralizados da seguinte forma:
  \begin{enumerate}
    \item A equipe de coleta entrega o lixo orgânico e o óleo para a equipe de tratamento orgânico
    \item Lixo orgânico úmido é tratado sendo colocado em um biodigestor
    \item O biogás gerado pelo biodigestor será utilizado em geradores de energia
    \item Gerador elétrico irá utilizar a energia gerada para ajudar na alimentação do próprio sistema de energia elétrica.
  \end{enumerate}

  O recebimento do lixo orgânico e o processamento do lixo pelo Biodigestor será feito todo pela empresa EcoLogistic. O fertilizante gerado será vendido a produtores rurais cujo serão beneficiados. O biogás produzido pelo biodigestor será convertido em energia elétrica e disponibilizado a concessionária de energia beneficiando a população.
