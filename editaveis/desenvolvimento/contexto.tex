\section{Contextualização}

  Atualmente, um dos maiores problemas que a humanidade enfrenta é a poluição do meio ambiente que leva à degradação deste, e consequentemente, a riscos para a própria humanidade. Riscos estes como a falta de água potável, destruição da camada de ozônio, degradação da saúde urbana, etc. Todas estas consequências têm levado a humanidade a repensar sobre suas atitudes e buscar métodos de combate para a poluição do ambiente, seja visual, sonora, do solo, do ar ou da água.
  \par
  Catalisadores nos carros modernos, materiais biodegradáveis, reciclagem, entre outros métodos, têm sido utilizados na busca de diminuir o efeito global da poluição. A sociedade moderna caminha rumo à sustentabilidade, buscando meios que garantam uma boa qualidade de vida para as futuras gerações, até porque a não concretização destes meios compromete a vida no nosso planeta.
  \par
  Apesar de se tratar de uma prática ilegal, o descarte inadequado de resíduos é muito comum no Brasil. Seja ele produto de uma falha na educação social ou irresponsabilidade, tem resultado em disseminação de doenças, contaminação do solo, das águas e do ar. O desenvolvimento de projetos que apresentam soluções para este problema é extremamente necessário para a garantia da qualidade de vida da sociedade brasileira. Neste cenário, este projeto planeja a adoção de uma coleta eficiente e um tratamento adequado do lixo urbano produzido em uma área predeterminada do Brasil.
  
